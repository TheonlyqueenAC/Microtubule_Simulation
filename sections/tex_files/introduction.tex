\begin{document} % ✅ This should be here, no blank lines before it.
\hypersetup{hidelinks, unicode=true, bookmarksopen=true, linkcolor=blue, citecolor=blue, urlcolor=blue}
\section{Introduction}
\subsection{Study Overview}
Microtubules, dynamic cytoskeletal structures, have been proposed as quantum-coherent biological systems capable of computational processes. However, Tegmark (2000) argued that quantum coherence in biological systems should decohere within femtoseconds due to environmental interactions, making quantum processing in the brain implausible. Contrary to this, emerging evidence suggests that structured environments such as microtubules may actively regulate and preserve coherence over biologically relevant timescales.
The challenge of studying quantum coherence in microtubules parallels similar challenges in astrophysical modeling, where direct empirical validation is often impractical. To address this, we employ Fibonacci scaling, a mathematical principle extensively used in astrophysical frameworks, to explore how coherence may be sustained in biological systems. This approach follows the precedent set in astrophysics, where mathematical solutions—though unobservable directly—offer predictive power and theoretical consistency. The frequent occurrence of Fibonacci scaling in biological structures further supports its applicability to quantum stability in microtubules.

Skepticism toward microtubular quantum coherence stems from Tegmark’s (2000) argument that thermal and molecular interactions should rapidly decohere any quantum states in biological systems. However, our computational models suggest that microtubules may sustain coherence longer than previously assumed by forming “quantum sanctuaries”—protected regions where coherence is stabilized through intrinsic mechanisms. Specifically, this work:
\begin{itemize} 
\item Integrates Fibonacci scaling, a universal mathematical pattern in biological systems, to demonstrate enhanced coherence stability.
\item Applies event-horizon analogies from astrophysics, proposing that microtubules create boundary-like structures that confine coherence and protect against decoherence.
\item Simulates wavefunction evolution under cytokine-mediated perturbations, revealing potential resilience mechanisms.
\end{itemize}
By introducing testable computational predictions, this study challenges assumptions about decoherence in biological systems, offering a new framework for understanding quantum stabilization in neural computation.

\subsection{Comparison with Existing Theories}
Several models have previously explored quantum coherence in microtubules:
\begin{itemize}
\item Kozłowski \& Marciak-Kozłowska (2005) analyzed quantum heat transport, suggesting that microtubules regulate biological processes through quantum effects.
\item Mershin et al. (2000) proposed that microtubules function as quantum computational networks, treating tubulin dimers as qubits for cognitive processing.
\item Issokolo et al. (2023) examined soliton propagation in microtubules, indicating that nonlinear oscillations may play a role in neural signaling and consciousness.
\end{itemize}
While these studies suggest that microtubules engage in quantum processes, they do not directly address how coherence is preserved in the presence of biological noise. This study extends previous work by demonstrating that Fibonacci scaling may act as a stabilizing factor that reduces wavefunction dispersion, thereby sustaining coherence despite cytokine-induced perturbations.

HIV-associated neurocognitive disorder (HAND) provides a well-defined model to study coherence loss due to chronic neuroinflammation. HAND is characterized by persistent immune activation, with elevated TNF-$\alpha$, IL-6, and IL-1$\beta$ driving progressive neuronal dysfunction. Additionally, HIV proteins such as Tat and gp120 directly disrupt neuronal cytoskeletal stability, accelerating microtubule degradation.

This study presents a computational model demonstrating how HIV-driven cytokine perturbations lead to progressive microtubule coherence loss. Furthermore, we introduce a novel event horizon framework, mathematically demonstrating that microtubules exhibit coherence-preserving boundary conditions analogous to astrophysical event horizons. By integrating quantum mechanics, neuroscience, and computational modeling, this work provides a direct computational challenge to Tegmark’s hypothesis and establishes a framework for understanding coherence regulation in disease.