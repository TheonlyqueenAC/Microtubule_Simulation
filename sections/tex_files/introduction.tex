?%--------------------?
?% Introduction?
?%--------------------?
?\section{Introduction}?
The development of the human fetal brain in late gestation represents a pivotal period in ?
neurodevelopment. By this stage, the number of neurons in the brain reaches its lifetime ?
maximum, alongside an intricate network of synaptic connections that set the stage for the ?
establishment of consciousness.?

Microtubules, dynamic cytoskeletal structures within neurons, are uniquely positioned to ?
play a role in this process. Beyond their well-documented roles in intracellular transport ?
and structural integrity, microtubules have been hypothesized to exhibit quantum ?
coherence, as proposed in the Orch-OR model by Hameroff and Penrose. However, ?
critiques, such as Tegmark's assertion of rapid decoherence in biological systems, ?
challenge the feasibility of sustained quantum states in microtubules.?

This study introduces computational models demonstrating the persistence of quantum ?
coherence in microtubules despite perturbations, including those induced by cytokines ?
such as IL-6, IL-1$\beta$, and TNF-$\alpha$. By employing a computational framework ?
grounded in the Schršdinger equation, the study extends microtubule dynamics to cosmic ?
scales using Fibonacci scaling.?


