\section{Discussion}
\subsection{Addressing Tegmark's Critique}
Tegmark (2000) argued that quantum states in biological systems decohere too rapidly to play a role in cognition. However, our findings challenge this claim by demonstrating that Fibonacci scaling and structured boundary conditions mitigate decoherence effects in microtubules. Unlike previous quantum brain theories, this study quantifies how coherence persists through self-organizing boundary effects, providing a computationally testable hypothesis for future experiments.
\begin{itemize}
    \item Persistence of coherence despite cytokine perturbations: Our HAND-driven simulations indicate that quantum coherence persists for biologically relevant timescales, even under sustained inflammatory perturbations.
    \item Fibonacci Scaling Enhances Stability: Probability density analysis of wavefunction evolution under cytokine stress revealed that Fibonacci-scaled microtubules maintained coherence longer than non-Fibonacci lattices.
    \item Event-horizon-like coherence boundaries: Simulations demonstrated that structured coherence-preserving boundaries emerge dynamically within microtubules, delaying decoherence.
\end{itemize}

\subsection{Distinction from Orch-OR and Other Models}
This study refines and extends Orch-OR by introducing a specific computational framework that accounts for coherence preservation mechanisms. Unlike previous models, which largely focused on qualitative descriptions, this approach provides mathematical and visual evidence supporting coherence stabilization.
\begin{table}[H]
\centering
\caption{Comparison of the Orch-OR Model and the Current Study.}
\label{tab:orch_or_comparison}
\begin{tabular}{|p{3.8cm}|p{3.8cm}|p{3.8cm}|}
\hline
\textbf{Feature} & \textbf{Orch-OR Model (Hameroff \& Penrose, 1996)} & \textbf{Current Study} \\
\hline
\textbf{Quantum Coherence in Microtubules} & Assumed but lacked a specific stabilizing mechanism & Demonstrated via Fibonacci scaling and event-horizon-like structures \\
\hline
\textbf{Decoherence Mechanisms} & External interactions cause rapid collapse & Protective zones (quantum sanctuaries) mitigate decoherence \\
\hline
\textbf{Computational Evidence} & Largely conceptual, minimal simulations & Explicit simulations of wavefunction evolution under cytokine-induced perturbations \\
\hline
\textbf{Scaling Considerations} & Assumed microtubules operate solely at neuronal levels & Integrated cosmic-scale mathematical principles (Fibonacci scaling) \\
\hline
\textbf{Novel Theoretical Contribution} & Proposes quantum processes in microtubules but lacked precise physical mechanisms & Introduces a testable model for coherence stabilization using boundary conditions and Fibonacci scaling \\
\hline
\end{tabular}
\end{table}

\subsection{HAND as a Model for Cytokine-Driven Coherence Loss}
HAND provides a biologically relevant case study for quantum coherence loss due to:
\begin{itemize}
    \item Well-characterized cytokine-mediated neuroinflammation.
    \item HIV-induced microtubule destabilization, allowing direct comparison to quantum models.
\end{itemize}

\subsubsection{Persistence of Coherence in Microtubules Despite Cytokine Perturbations}
Our HAND-driven simulations indicate that quantum coherence persists for biologically relevant timescales, even under sustained inflammatory perturbations. In particular:
\begin{itemize}
    \item In early-stage HAND, coherence remained stable despite elevated TNF-$\alpha$ and IL-6 levels. This contradicts Tegmark's claim that quantum coherence in biological tissue should decohere within femtoseconds.
    \item Coherence degradation was non-instantaneous and correlated with inflammatory load. Wavefunction evolution revealed progressive but non-exponential collapse, indicating an active biological regulation process rather than passive decoherence.
    \item Regions of protected coherence, analogous to event horizons, emerged dynamically within the microtubule lattice. These "quantum sanctuaries" exhibited coherence stability even as surrounding regions degraded.
\end{itemize}

\subsection{Fibonacci Scaling as a Universal Quantum Stabilization Mechanism}
The use of Fibonacci scaling in this study is not arbitrary but follows a logical extension of its application in astrophysics, where it has provided robust mathematical solutions to phenomena that remain empirically unobservable, such as event horizon boundary dynamics. Given that empirical measurement of microtubular quantum coherence is currently beyond available technology, we employ Fibonacci scaling as a mathematical framework to explore potential stabilizing mechanisms that would otherwise be inaccessible through direct experimentation. This approach is consistent with methodologies in astrophysics, where mathematical models—though lacking direct empirical verification—are widely accepted when they (1) adhere to fundamental physical laws, (2) exhibit internal consistency, and (3) produce predictions that align with indirect observations. 

Fibonacci scaling has been widely observed in biological structures, and our findings suggest that:
\begin{itemize}
\item Microtubular structures exhibit Fibonacci resonance patterns that reduce wavefunction dispersion.
\item Probability density analysis of wavefunction evolution under cytokine stress revealed that Fibonacci-scaled microtubules maintained coherence longer than non-Fibonacci lattices.
\item Fibonacci-derived coherence stabilization provides an alternative mechanism for quantum persistence beyond traditional decoherence models.
\end{itemize}

These findings provide a mathematical and computationally testable argument against Tegmark's prediction of rapid quantum decoherence in biological systems. While the direct empirical validation of quantum sanctuaries in microtubules remains an open challenge, this study provides a computationally rigorous framework that allows for testable predictions, analogous to the way astrophysical models advance understanding of black holes without requiring direct observational evidence. Future advances in quantum biological measurement techniques may offer opportunities to validate these findings.

\subsubsection{Implications for Quantum Biology and Neurodegeneration}
These results suggest that quantum coherence:
\begin{itemize}
    \item Is not instantly destroyed in biological environments, contradicting Tegmark's femtosecond-scale decoherence argument.
    \item Can be dynamically regulated by structured cellular environments, particularly through Fibonacci scaling and coherence boundary formation.
    \item May be selectively degraded under neuroinflammatory conditions, providing a quantum framework for understanding neurodegenerative diseases like HAND.
\end{itemize}

\subsection{Decline of Consciousness in HAND}
The progressive cognitive decline observed in HAND can be conceptualized as the gradual breakdown of quantum coherence within microtubules, leading to a fragmentation of integrated neural processing. While synaptic networks provide the structural architecture for cognition, it is the persistence of quantum coherence within microtubules that may enable large-scale integration of information—an essential feature of conscious awareness. 

Our computational findings provide a novel perspective on this process, demonstrating that as cytokine-induced perturbations disrupt microtubular coherence, the brain's ability to maintain quantum-integrated processing diminishes. In early-stage HAND, coherence is partially preserved despite increasing neuroinflammation, mirroring the mild cognitive impairments seen in People living with HIV. However, as cytokine exposure intensifies and coherence loss accelerates, microtubules transition from a stable quantum state to a progressively disordered one, leading to fragmentation of cognitive function.

This study proposes that event horizon-like boundaries within microtubules regulate coherence persistence, and their collapse under sustained neuroinflammation correlates with the progressive loss of consciousness observed in late-stage HAND. In this framework, the breakdown of microtubule coherence is not merely a symptom of neurodegeneration but may be directly implicated in the fundamental degradation of conscious experience itself. 

These results suggest a quantum-informed approach to understanding neurocognitive disorders, where diseases like HAND can be studied as progressive decoherence phenomena, providing a bridge between quantum mechanics and consciousness research.