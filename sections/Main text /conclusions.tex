\section{Conclusion}
\subsection{Key Contributions}
\begin{itemize}
\item Proposes "Event Horizon Analogies" as stabilizing regions in microtubules. Unlike existing theories, this study provides a quantifiable model for coherence protection in biological systems.
 \item Integrates Fibonacci Scaling into Quantum Biology, introducing self-organizing scaling laws as a stabilizing force against decoherence.
\item Provides a direct computational challenge to Tegmark's decoherence hypothesis, demonstrating that:
\begin{enumerate}
\item Quantum coherence persists in biological microtubules despite cytokine perturbations.
\item HIV-driven inflammation selectively degrades coherence, validating a structured, disease-driven decoherence model.
\item The event horizon framework suggests that microtubules may regulate coherence boundaries dynamically.
\end{enumerate}
\end{itemize}

These findings reshape our understanding of coherence loss in disease and provide a new framework for investigating quantum biology in neurodegenerative conditions.

\subsection{Demonstrated Computational Coherence Persistence} 
Simulations show that wavefunction coherence persists even under cytokine-induced perturbations, countering previous claims of rapid decoherence.

\subsection{Limitations}
Despite its strong theoretical foundation, this study acknowledges several key limitations:
\begin{itemize}
    \item \textbf{Lack of Experimental Data:} Although the findings are compelling, empirical studies are needed to confirm microtubule coherence persistence in biological systems.
    \item \textbf{Simplified Mathematical Models:} The study uses idealized wavefunctions, which may not fully capture biological complexity.
    \item \textbf{Environmental Effects:} The impact of thermal noise, molecular interactions, and biological fluctuations on coherence stabilization requires further investigation.
\end{itemize}

\subsection{Future Research Directions}
\begin{itemize}
    \item Development of experimental models to validate Fibonacci-driven coherence stabilization in microtubules.
    \item Investigation of quantum measurement techniques to detect coherence persistence in biological systems.
    \item Apply this model to other neurodegenerative diseases (e.g. Alzheimer's, Parkinson's).
    \item Investigate potential therapeutic interventions targeting coherence preservation.
    \item Exploration of artificially engineered quantum cognitive systems for potential applications in quantum computing and neurotechnology.
\end{itemize}

\subsection{Final Remarks}  
This study establishes a computational framework for investigating quantum coherence in biological systems using well-established astrophysical principles. Although direct empirical verification remains challenging, the predictive power of Fibonacci scaling in stabilizing coherence offers a testable hypothesis. Future advancements in quantum biology measurement techniques could lead to the experimental validation of these findings, marking a significant step forward in understanding the intersection of quantum mechanics, biology, and consciousness.

