%  LaTeX support: latex@mdpi.com 
%  For support, please attach all files needed for compiling as well as the log file, and specify your operating system, LaTeX version, and LaTeX editor.
%=================================================================
\documentclass[entropy,article,submit,oneauthor,pdftex]{Definitions/mdpi}

\usepackage{etoolbox} % Ensure compatibility with document class
\usepackage[utf8]{inputenc}  % Enables Unicode support
\usepackage{textcomp}  % Provides additional symbols
\usepackage{newunicodechar}  
\usepackage[sort&compress,sectionbib]{natbib} % For bibliography handling
\newunicodechar{α}{$\alpha$}
\newunicodechar{β}{$\beta$}
\usepackage{float} % Ensure proper float handling
\usepackage{placeins} % Improved float placement
\setlength{\headheight}{24.18796pt}
\addtolength{\topmargin}{-12.18796pt}
\usepackage[unicode=true, bookmarksopen={true}, pdffitwindow=true, colorlinks=true, linkcolor=bluecite, citecolor=bluecite, urlcolor=bluecite, hyperfootnotes=true, pdfstartview={FitH}, pdfpagemode=UseNone]{hyperref}
% MDPI internal commands - do not modify
\firstpage{1} 
\makeatletter 
\setcounter{page}{\@firstpage} 
\makeatother
\pubvolume{1}
\issuenum{1}
\articlenumber{0}
\pubyear{2025}
\copyrightyear{2025}
\datereceived{February 26, 2025} 
\daterevised{} % Comment out if no revised date
\dateaccepted{} 
\datepublished{} 
\hreflink{https://doi.org/TBD}  % Replace with actual DOI when available
\usepackage{booktabs}
\usepackage{orcidlink}
\usepackage{cite}
\usepackage{textcomp}
\usepackage{gensymb}
\usepackage{upgreek}
\usepackage{caption}
\usepackage{siunitx}
\hypersetup{hidelinks, unicode=true, 
bookmarksopen=true, 
colorlinks=true, linkcolor=blue, 
citecolor=blue, urlcolor=blue, 
pdfpagemode=UseNone}

% Full title of the paper (Capitalized)
\Title{DECOHERENCE, DISEASE, AND THE QUANTUM BRAIN: HIV-DRIVEN NEUROINFLAMMATION AS A MODEL FOR QUANTUM DECOHERENCE IN MICROTUBULES}
\author{AC Demidont DO \orcidlink{0000-0002-9216-8569}}
\address{Nyx Dynamics LLC, 268 Post Rd, Ste 200, Fairfield, CT 06824, USA}
\corres{Correspondence:acdemidont@nyxdynamics.org}

\begin{document}
\maketitle

\begin{abstract}
Quantum coherence, a phenomenon typically considered fragile in biological systems, is widely assumed to decohere rapidly under physiological conditions. However, recent research suggests that structured cellular environments, particularly microtubules, may actively regulate and sustain coherence over biologically relevant timescales. Tegmark \cite{tegmark_importance_2000} famously argued that quantum states in biological systems should decohere within femtoseconds, making quantum processing in the brain implausible. This study challenges this assumption by demonstrating that microtubules, when subjected to biologically realistic perturbations, exhibit structured coherence decay rather than immediate collapse. Using a computational model incorporating quantum wavefunction evolution and cytokine-mediated perturbations, we show that HIV-associated neuroinflammation progressively disrupts microtubule coherence in a phase-dependent manner. Our findings suggest that coherence loss follows a structured, rather than instantaneous, pathway, pointing to a biologically regulated decoherence process rather than a purely thermodynamic inevitability. Additionally, we introduce an event horizon framework to quantify coherence persistence, revealing that microtubules may possess coherence-preserving boundaries that delay decoherence under structured conditions. These results have implications for neuroscience, artificial intelligence, and quantum cognition, providing a testable framework for understanding how biological systems can actively regulate coherence stability. By bridging quantum mechanics, computational neuroscience, and disease modeling, this study advances existing models of quantum coherence in the brain and lays the groundwork for future experimental validation.
\end{abstract}

\keywords{Microtubules; Quantum Coherence; Decoherence; Fibonacci Scaling; Event Horizons; Neural Computation; Cytokine Perturbations; Quantum Biology; HIV-Associated Neurocognitive Disorder; Neuroinflammation; Consciousness; Artificial Intelligence; Computational Neuroscience; Neurotechnology} 

