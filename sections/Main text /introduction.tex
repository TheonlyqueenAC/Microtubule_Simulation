\section{Introduction}
\subsection{Study Overview}
Microtubules have long been hypothesized to support quantum processes within biological systems, particularly in the context of consciousness, as proposed by the Orch OR theory \cite{hameroff_orchestrated_1996,nanopoulos_quantum_1995}. Tegmark's decoherence argument \cite{tegmark_importance_2000} challenged this view, contending that quantum coherence in biological environments would collapse within femtoseconds due to thermal and molecular interactions. However, emerging research suggests that microtubules possess structural and energetic properties that may allow them to maintain coherence for significantly longer durations than previously assumed.

Experimental studies have demonstrated that tubulin contains chromophores capable of supporting long-range quantum coherence, similar to photosynthetic light harvesting systems \cite{craddock_feasibility_2014}. Furthermore, microtubules have been proposed as quantum electrodynamical cavities, exhibiting conditions conducive to coherence stabilization \cite{mavromatos_quantum_2011}. These findings suggest that biological systems may have evolved mechanisms to protect coherence as a fundamental aspect of cognition. Recent advances in quantum phase transitions suggest that finite-size effects influence the lifetime of coherence \cite{pelissetto_scaling_2023}. In microtubules, these effects may stabilize coherence through boundary conditions, an argument that aligns with our Fibonacci scaling model.

The challenge of studying quantum coherence in microtubules parallels similar challenges in astrophysical modeling, where direct empirical validation is often impractical. To address this, we employ Fibonacci scaling, a mathematical principle extensively used in astrophysics, to explore how coherence may be sustained in biological systems. Fibonacci scaling is known to enhance stability in non-linear dynamical systems by optimizing resonance conditions and reducing wave function dispersion. Its frequent occurrence in biological structures suggests an underlying role in the stabilization of coherence.

\subsection{Comparison with Existing Theories}
Several models have previously explored quantum coherence in microtubules:
\begin{itemize}
\item Kozłowski \& Marciak-Kozłowska analyzed quantum heat transport, suggesting that microtubules regulate biological processes through quantum effects \cite{kozlowski_wave-gtdiffusion_2005}.
\item Mershin et al. proposed that microtubules function as quantum computational networks, treating tubulin dimers as qubits for cognitive processing \cite{mershin_quantum_2000}.
\item Issokolo et al. examined soliton propagation in microtubules, indicating that nonlinear oscillations may play a role in neural signaling and consciousness \cite{issokolo_localized_2024}.
\end{itemize}

Although these studies suggest that microtubules engage in quantum processes, they do not directly address how coherence is preserved in the presence of biological noise. This study extends previous work by computationally demonstrating that Fibonacci scaling acts as a stabilizing factor that reduces wavefunction dispersion, thus sustaining coherence even under cytokine-induced perturbations.

\subsection{HIV-Associated Neurocognitive Disorder as a Model for Coherence Collapse}
HIV-associated neurocognitive disorder (HAND) provides an ideal model to study coherence loss due to chronic neuroinflammation. HAND is characterized by persistent immune activation, with elevated TNF-$\alpha$, IL-6, and IL-1$\beta$ driving progressive neuronal dysfunction \cite{zhou_development_2025}. Furthermore, HIV proteins such as Tat and gp120 directly alter neuronal cytoskeletal stability, accelerating microtubule degradation \cite{thompson_hiv-associated_2024}. This aligns with the findings that latent HIV reservoirs in microglia contribute to neuroinflammation and progressive neuronal damage \cite{sreeram_potential_2022}. Studies have shown that HAND remains prevalent despite antiretroviral therapy (ART), suggesting that latent HIV activity and inflammatory cytokines play a major role in the continued neurodegenerative process \cite{thompson_hiv-associated_2024}.

Unlike previous studies that examined the loss of coherence in hypothetical biological settings, this study presents a computational model that demonstrates how HIV-driven cytokine perturbations lead to a progressive loss of microtubule coherence. Furthermore, we introduce a novel event horizon framework, mathematically demonstrating that microtubules exhibit coherence-preserving boundary conditions analogous to astrophysical event horizons. By integrating quantum mechanics, neuroscience, and computational modeling, this work provides a direct computational challenge to Tegmark's hypothesis and establishes a framework for understanding coherence regulation in disease.

%%%%%%%%%%%%%%%%%%%%%%%%%%%%%%%%%%%%%%%%%%